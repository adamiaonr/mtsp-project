\section{Introduction}
\label{sec:intro}

Departing from its initial model as a network for host-to-host communications, 
the Internet started shifting towards a content-centric model with the advent of 
the World Wide Web in the 1980s. This model persisted, 
and with increasingly demanding usage requirements, leading to the 
development of technologies such as 
Content Delivery Networks (CDNs) and Peer-to-Peer (P2P) 
networks~\cite{Passarella20121}. These were built around the architecture's 
edge, due to the so-called 
`ossification'~\cite{Handley:2006:WIO:1188052.1188101} of the Internet's core, 
leading to inefficiencies in terms of latency, bandwidth usage, among others. 
Given the widespread adoption of the content-centric model, 
researchers to think about new and clean-slate designs 
for the Internet's core, in order for it to natively cope with these issues. Among 
such efforts~\cite{5936152}, the research field of Information Centric Networking 
(ICN)~\cite{Xylomenos2013} emerged, 
advocating the deliberate abolition of network locators, replacing of IP 
addresses with content identifiers and calling for the widespread use of 
in-network caching, so that content can be easily served from multiple 
anywhere in the network~\cite{Koponen2007,Jacobson2009,Trossen2012,
Raychaudhuri2012,Han2012,Dannewitz:2013:NII:2459510.2459643}. Here we focus on the aspect of 
in-network caching in one of such clean slate designs, 
the Named Data Networking (NDN) architecture~\cite{Jacobson2009}.\shortvertbreak

In this paper, we focus on the analysis of cache behavior in NDN 
networks under different cache algorithms, network topologies and content 
usage characteristics. To do so, we specify a simple and but modular NDN router 
model, loosely inspired in nonlinear dynamical 
systems~\cite{Hedrick2010}. We implement the specified model in MATLAB\textsuperscript{\textregistered}, providing 
some simulation results with three simple cache algorithms, specifically (1) Least Recently 
Used (LRU), (2) More Recently Used (MRU) and (3) Random caching. The main contribution 
of this work consists in the provision of a framework in MATLAB\textsuperscript{\textregistered}, 
which allows for the simulation of NDN network behavior under different 
topologies, cache algorithms, NDN router characteristics, etc. without 
the complexity of more elaborated network simulators.\shortvertbreak

The remainder of this paper is organized as follows. In Section~\ref{sec:ndn} we provide an 
overview over the NDN architecture, focusing on the basic operation of its 
forwarding engine and the way it involves in-network caching. In 
Section~\ref{sec:methodology}, we present the overall methodology followed 
during this work, including an explanation of the considered NDN router model, 
network topologies to be considered, cache algorithms, etc. In Section~\ref{sec:experiments} we present a set of 
experiments ran over our model implementation, as well as the respective 
results. Finally, in Section~\ref{sec:conclusions} we draw some pertinent 
conclusions from the presented work.
