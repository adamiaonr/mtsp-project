\section{Conclusions}
\label{sec:conclusions}

In this work, we have presented the specification details of an Named Data 
Networking (NDN)~\cite{Jacobson2009} simulation model, inspired by the field of nonlinear dynamical systems, which 
allows the evaluation of different caching policies under different topologies. To 
do so, we have understood the mechanics of NDN's forwarding engine and proposed 
models for each of its components, identifying the relevant types of inputs, 
outputs, state and associated nonlinear dynamics.\shortvertbreak

To demonstrate the feasibility of the proposed model, we implemented the 
given specifications in MATLAB\textsuperscript{\textregistered}, providing 
simulation results with three simple cache algorithms, specifically (1) Least Recently 
Used (LRU), (2) More Recently Used (MRU) and (3) Random caching. As the purpose 
of this work was to assess the basic feasibility of the proposed model, we 
considered simplistic topologies, similar to those considered in NDN analytical 
models such as~\cite{6038471,Psaras:2011:MEC:2008780.2008789,Psaras2012,Chai2013}. As 
future work, we intend to test the model under more complex topologies --- e.g. 
those considered in a similar work~\cite{6386648} --- and novel cache algorithms.\shortvertbreak

The results 
obtained after simulation allow one to identify some of the key features claimed 
by the NDN architecture~\cite{Jacobson2009} --- in a way asserting the validity 
of the proposed model --- as well as specificities of 
the considered cache algorithms. We believe the proposed model specification 
and implementation to be of particular interest for designers wishing to 
perform preliminary evaluations of caching policies under NDN. Even though the 
purpose of the proposed model is not to strive for a truly realistic 
scenario, as future enhancements we predict the introduction of packet loss, 
network congestion, among other parameters, in order to make it more robust.
